% Documento LaTex com o artigo que estamos escrevendo

% Cabeçalho
%Onde a gente configura o documento
%%%%%%%%%%%%%%%%%%%%%%%%%%%%
\documentclass{article}

\usepackage[brazil]{babel}
\usepackage{graphicx}
\usepackage [round,authoryear,sort]{natbib}
\usepackage{mathpazo}
\usepackage{notomath}

\newcommand{\Title}{Análise de variação de temperatura dos últimos cinco (5) anos}
\input{paises.tex}



% Corpo
% Onde a gente escreve o texto
%%%%%%%%%%%%%%%%%%%%%%%%%%%%%%%%%%%%%%%%
\begin{document}

\title{Análise de variação de temperatura dos últimos cinco anos}
\author{Osvaldo Pilarte}

\maketitle

\begin{abstract}
Meu resumo
\end{abstract}

Meu artigo bem legal

\section{Introdução}

Isso vai ser a minha introdução
Outra frase da introduça


Esse já será outro parágrafo da introdução.

Trabalhos anteriores bem legais fizeram coisas parecidas parecidas
\citep{Tohver_2006}.
Isso foi analisado primeiro por \citep{Tohver_2006}.

\section{Metodologia}

Aqui eu vou descrever tudo que eu fiz.
Ajustamos uma recta aos cinco últimos anos dos dados
de temperatura média mensal para cada país.
Assim calculamos a taxa de variação da temperatura recente.

\begin{equation}
y=\int\Omega x dx
\end{equation}

A equação da reta é

\begin{equation}
T(t) = a t + b
\label{eq:divina}
\end{equation}

\noindent

Utilizamos a equação \ref{eq:divina} em um código python para fazer o ajuste da recta com o método dos mínimos quadrados.
onde $T$ é a temperatura, $t$ é o tempo, $a$ é o coeficiente angular e $b$ é o coeficiente linear.

\section{Resultados}

Analisamos os dados de 225 países.
Os paises analisados foram: \Paises.



\begin{figure}
\includegraphics{../figuras/variacao_temperatura.png}
\caption{
    Variação de temperatura média mensal dos cinco 
    (5) últimos anos.
    a) Países com as cinco (5) maiores variações de temperatura.
    b) Países com as cinco (5) menores variações de temperatura.
}

\label{fig:variação}
\end{figure}

\bibliographystyle{apalike}
\bibliography{referencias.bib}

\end{document}    