% Documento LaTex com o artigo que estamos escrevendo

% Cabeçalho
%Onde a gente configura o documento
%%%%%%%%%%%%%%%%%%%%%%%%%%%%
\documentclass{article}

\usepackage[brazil]{babel}
\usepackage{graphicx}
\usepackage [round,authoryear,sort]{natbib}
\usepackage{mathpazo}
\usepackage{notomath}

\newcommand{\Title}{Análise de variação de temperatura dos últimos cinco (5) anos}
\newcommand{\Paises}{Afghanistan; Albania; Algeria; American Samoa; Andorra; Angola; Anguilla; Antarctica; Antigua And Barbuda; Argentina; Armenia; Aruba; Australia; Austria; Azerbaijan; Bahamas; Bahrain; Baker Island; Bangladesh; Barbados; Belarus; Belgium; Belize; Benin; Bhutan; Bolivia; Bonaire, Saint Eustatius And Saba; Bosnia And Herzegovina; Botswana; Brazil; British Virgin Islands; Bulgaria; Burkina Faso; Burma; Burundi; Cambodia; Cameroon; Canada; Cape Verde; Cayman Islands; Central African Republic; Chad; Chile; China; Christmas Island; Colombia; Comoros; Congo; Congo (Democratic Republic Of The); Costa Rica; Croatia; Cuba; Cyprus; Czech Republic; Denmark; Djibouti; Dominica; Dominican Republic; Ecuador; Egypt; El Salvador; Equatorial Guinea; Eritrea; Estonia; Ethiopia; Falkland Islands (Islas Malvinas); Faroe Islands; Fiji; Finland; France; French Guiana; French Polynesia; French Southern And Antarctic Lands; Gabon; Gambia; Gaza Strip; Georgia; Germany; Ghana; Greece; Greenland; Grenada; Guadeloupe; Guatemala; Guernsey; Guinea; Guinea Bissau; Guyana; Haiti; Heard Island And Mcdonald Islands; Honduras; Hong Kong; Hungary; Iceland; India; Indonesia; Iran; Iraq; Ireland; Isle Of Man; Israel; Italy; Jamaica; Japan; Jersey; Jordan; Kazakhstan; Kenya; Kingman Reef; Kuwait; Kyrgyzstan; Laos; Latvia; Lebanon; Lesotho; Liberia; Libya; Liechtenstein; Lithuania; Luxembourg; Macau; Macedonia; Madagascar; Malawi; Malaysia; Mali; Malta; Martinique; Mauritania; Mauritius; Mayotte; Mexico; Moldova; Monaco; Mongolia; Montenegro; Montserrat; Morocco; Mozambique; Namibia; Nepal; Netherlands; New Caledonia; New Zealand; Nicaragua; Niger; Nigeria; Niue; North Korea; Northern Mariana Islands; Norway; Oman; Pakistan; Palestina; Palmyra Atoll; Panama; Papua New Guinea; Paraguay; Peru; Philippines; Poland; Portugal; Puerto Rico; Qatar; Reunion; Romania; Russia; Rwanda; Saint Kitts And Nevis; Saint Lucia; Saint Martin; Saint Pierre And Miquelon; Saint Vincent And The Grenadines; Samoa; San Marino; Sao Tome And Principe; Saudi Arabia; Senegal; Serbia; Seychelles; Sierra Leone; Singapore; Sint Maarten; Slovakia; Slovenia; Solomon Islands; Somalia; South Africa; South Georgia And The South Sandwich Isla; South Korea; Spain; Sri Lanka; Sudan; Suriname; Svalbard And Jan Mayen; Swaziland; Sweden; Switzerland; Syria; Taiwan; Tajikistan; Tanzania; Thailand; Timor Leste; Togo; Tonga; Trinidad And Tobago; Tunisia; Turkey; Turkmenistan; Turks And Caicas Islands; Uganda; Ukraine; United Arab Emirates; United Kingdom; United States; Uruguay; Uzbekistan; Venezuela; Vietnam; Virgin Islands; Western Sahara; Yemen; Zambia; Zimbabwe}




% Corpo
% Onde a gente escreve o texto
%%%%%%%%%%%%%%%%%%%%%%%%%%%%%%%%%%%%%%%%
\begin{document}

\title{Análise de variação de temperatura dos últimos cinco anos}
\author{Osvaldo Pilarte}

\maketitle

\begin{abstract}
Meu resumo
\end{abstract}

Meu artigo bem legal

\section{Introdução}

Isso vai ser a minha introdução
Outra frase da introduça


Esse já será outro parágrafo da introdução.

Trabalhos anteriores bem legais fizeram coisas parecidas parecidas
\citep{Tohver_2006}.
Isso foi analisado primeiro por \citep{Tohver_2006}.

\section{Metodologia}

Aqui eu vou descrever tudo que eu fiz.
Ajustamos uma recta aos cinco últimos anos dos dados
de temperatura média mensal para cada país.
Assim calculamos a taxa de variação da temperatura recente.

\begin{equation}
y=\int\Omega x dx
\end{equation}

A equação da reta é

\begin{equation}
T(t) = a t + b
\label{eq:divina}
\end{equation}

\noindent

Utilizamos a equação \ref{eq:divina} em um código python para fazer o ajuste da recta com o método dos mínimos quadrados.
onde $T$ é a temperatura, $t$ é o tempo, $a$ é o coeficiente angular e $b$ é o coeficiente linear.

\section{Resultados}

Analisamos os dados de 225 países.
Os paises analisados foram: \Paises.



\begin{figure}
\includegraphics{../figuras/variacao_temperatura.png}
\caption{
    Variação de temperatura média mensal dos cinco 
    (5) últimos anos.
    a) Países com as cinco (5) maiores variações de temperatura.
    b) Países com as cinco (5) menores variações de temperatura.
}

\label{fig:variação}
\end{figure}

\bibliographystyle{apalike}
\bibliography{referencias.bib}

\end{document}    